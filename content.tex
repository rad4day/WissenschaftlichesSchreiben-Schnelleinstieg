\section{Einleitung}
\label{sec:Einleitung}

Wissenschaftliche Artikel sind die Grundlage für die Dokumentation und den Austausch von Herausforderungen, Methoden, Ansätzen und Ergebnissen unter Forschenden. 
Das präzise und nachvollziehbare Schreiben soll deshalb im Studium im Rahmen von Seminar- und Abschlussarbeiten geübt werden. 
Dieses Dokument bietet einen Schnelleinstieg in die wichtigsten Grundlagen und Werkzeuge und ergänzt die ausführlichen Informationen des \infoLink{https://sdqweb.ipd.kit.edu/wiki/Wissenschaftliches_Schreiben}{SDQ-Wikis}.


\section{Was macht Schreiben wissenschaftlich?}
\label{sec:WissenschaftlichesSchreiben}

Wissenschaftliches Schreiben ist präzise, nüchtern, nachvollziehbar und überprüfbar. 
Irreführende oder umständlich formulierte Aussagen sind genauso zu vermeiden wie Spannungsbögen oder Übertreibungen (z.B. \enquote{[...] ist ein herausragendes Ergebnis.}). 
Wegen der oft komplexen Inhalte wird besonderen Wert auf eine einfache Beschreibung gelegt.

Gute wissenschaftliche Artikel sind sauber strukturiert und folgen dem \emph{Prinzip der Pyramide}: 
Inhalte werden zunächst stark abstrahiert dargestellt und erst später im Detail erklärt, was die Verständnis des Lesers erhöhen soll. 
Zudem weisen die meisten Papiere eine ähnliche Struktur auf: Zusammenfassung, Einleitung, Grundlagen, Hauptkapitel (z.B. der eigene Ansatz), Evaluation und Fazit. 
Vor dem Fazit oder nach den Grundlagen wird Bezug auf verwandte Arbeiten genommen, da Forschung selten isoliert statt findet. 
Jedes einzelne Kapitel wiedrum sollte mit einem kurzen Überblick beginnen, gefolgt von detaillierteren Informationen. 
Zu Zusammenfassung, Einleitung und Fazit gibt es darüber hinaus feste Vorgaben zu Inhalten und Reihenfolge. 
Hierzu:

\smallskip
\begin{itemize}[label={\symbolInfo}]
    \item \href{https://sdqweb.ipd.kit.edu/wiki/Wissenschaftliches_Schreiben}{SDQ-Wiki --- Wissenschaftliches Schreiben }
    \item \href{https://sdqweb.ipd.kit.edu/wiki/Wissenschaftliches_Schreiben/Stil}{SDQ Wiki --- Wissenschaftliches Schreiben/Stil}
    \item \href{https://sdqweb.ipd.kit.edu/wiki/Wissenschaftliches_Schreiben/Aufbau}{SDQ Wiki --- Wissenschaftliches Schreiben/Aufbau}
\end{itemize}


\section{Recherchieren}
\label{sec:Recherchieren}

Die Recherche, also das Suchen, Lesen und Zitieren anderer wissenschaftlicher Veröffentlichungen, gehört zu den Kernaufgaben wissenschaftlicher Arbeit.

\subsection{Richtig suchen}
\label{sec:Recherchieren:Suchen}

Üblicherweise werden zu Beginn einer Seminar- oder Abschlussarbeit einige Quellen von den Betreuenden zur Verfügung gestellt. 
Darüber hinaus sollen immer auch weitere Quellen und verwandte Arbeit gesucht werden. 
Zu den bekanntesten Werkzeugen hierfür zählen:

\smallskip
\begin{itemize}[label={\symbolTool}]
    \item \href{https://scholar.google.com/}{Google Scholar}, Suchmaschine für wissenschaftliche Veröffentlichungen aller Art
    \item \href{https://dblp.org/}{dblp}, eine umfassende Bibliographie der Informatik
    \item \href{https://citeseerx.ist.psu.edu/}{CiteSeer}, eine große und bekannte Zitationsdatenbank 
\end{itemize}
\smallskip

\noindent
Mögliche Suchbegriffe sind (Kombinationen von) Titeln, Stichwörtern, Autoren und Konferenzen. 
Diese können auch aus bereits vorhandener Literatur entnommen werden. 
Eine weitere Möglichkeit ist das Folgen von Referenzen in vorliegender Literatur. 
Suchmaschinen wie \toolLink{https://scholar.google.com/}{Google Scholar} ermöglichen zudem die Anzeige anderer Veröffentlichungen, die eine ausgewählte Quelle ebenfalls referenziert haben (\enquote{zitiert von: [...]}). 
Viele Papiere sind frei verfügbar (\emph{Open Access}) oder lassen sich aus dem Uni-Netz bzw. über einen \toolLink{https://www.scc.kit.edu/dienste/vpn.php}{VPN} kostenlos abrufen.

Nicht alle Quellen haben die gleiche Wertigkeit. 
Präferiert sollte nach Journal- oder Konferenz-Artikeln (etwa von hochwertigen Publishern wie \emph{IEEE} oder \emph{ACM}) gesucht werden, da diese, wie auch Workshop-Papiere \emph{peer-reviewed} sind. 
Diese unabhängige Begutachtung durch andere Wissenschaftler gilt als wichtige Maßnahme der Qualitätssicherung. 
Zu Quellenarten ohne Peer-Review zählen technische Berichte, Dissertationen, Abschlussarbeiten oder Papiere in e-Print-Archiven wie \href{https://arXiv.org}{arXiv.org}.  
\href{https://www.wikipedia.org/}{Wikipedia} eignet sich nur fur einen initialen Überblick.

\subsection{Richtig lesen}
\label{sec:Recherchieren:Lesen}

Papiere sollten stets mit einem Ziel oder definierten Fragen gelesen werden. 
Um herauszufinden, ob ein gefundenes Papier relevant ist, reicht es nicht aus, nur den Titel zu lesen. 
Auf der anderen Seite enthalten wissenschaftliche Veröffentlichungen eine Vielzahl von Details, die zur Beantwortung der eigenen Fragen nicht notwendig sind. 
Aus diesen Gründen sollten Papiere \emph{zielgerichtet} gelesen werden: Zunächst Zusammenfassung, Gliederung, Einleitung und Zusammenfassung, und erst dann relevante Abschnitte.

Der Inhalt anderer Papiere sollte immer kritisch hinterfragt werden: Was wurde warum und wie gemacht, wie lassen sich Vorgehen, Ergebnisse und Evaluation bewerten? 
Erkenntnisse und eigene Kommentare können mit farblichen Markierungen festgehalten werden, etwa für Kategorien wie: \emph{Relevant}, \emph{Unklar} oder \emph{Interessante Referenz}.

\subsection{Richtig zitieren}
\label{sec:Recherchieren:Zitieren}

Alle Inhalte, die aus anderen Arbeiten stammen, müssen als Zitat gekennzeichnet werden. 
In wissenschaftlichen Arbeiten werden Zitate üblicherweise an Ort und Stelle markiert und dann im Literaturverzeichnis am Ende des Dokuments aufgeführt. 
Direkte Zitate sind wörtlich übernommen und werden in Anführungszeichen gefasst, indirekte Zitate werden z.B. am Ende der inhaltlich übernommenen Aussage markiert. 
Das Erscheinungsbild der Markierungen ist meist vorgegeben, das \emph{IEEE} empfiehlt etwa Zahlen in eckigen Klammern \cite{IEEE2021}. 
Hierzu:

\smallskip
\begin{itemize}[label={\faInfoCircle}]
    \item \href{https://sdqweb.ipd.kit.edu/wiki/Zitieren}{SDQ-Wiki --- Zitieren}
\end{itemize}


\section{Schreiben}
\label{sec:Schreiben}

\subsection{Die Sprache \LaTeX}

Kein WYSIWYG

https://sdqweb.ipd.kit.edu/wiki/LaTeX

\subsection{Installation}
Installation
TexLive
Dateien und Strukturen

\subsection{Werkzeuge}

VS-Code mit Tex Workshop
Overleaf
LatexMK
stack exchange latex



\subsection{Bibtex vs. Biblatex}

Biber

\subsection{Vorlagen}

Dieses Dokument: IEEE Paper
Dokumentvorlagen - SDQ Wiki (kit.edu)

\subsection{Hilfreiche Pakete}

\subsection{Diagramme}

Diagramm editoren
Direkt in Latex

\section{Präsentieren}

Stil, frei und flüssig
Latex vs. PowerPoint

\section{Verwalten}

\subsection{Literaturverwaltung}

Zotero
Funktionalität

\subsection{Versionsverwaltung}

SVN, git

\section{Weitere Tipps und Tools}

Sprach-Tools nutzen, z.B. Linguee und DeepL
PDF in Word öffnen um Grammatik zu prüfen