\section{Einleitung}%
\label{sec:Einleitung}

Wissenschaftliche Artikel sind die Grundlage für die Dokumentation und den Austausch von Herausforderungen, Methoden, Ansätzen und Ergebnissen unter Forschenden.
Das präzise und nachvollziehbare Schreiben soll deshalb im Studium im Rahmen von Seminar- und Abschlussarbeiten geübt werden.
Dieses Dokument bietet einen Schnelleinstieg in die wichtigsten Grundlagen und Werkzeuge und ergänzt die ausführlichen Informationen des \infoLink{https://sdqweb.ipd.kit.edu/wiki/Wissenschaftliches_Schreiben}{SDQ-Wikis}.


\section{Was macht Schreiben wissenschaftlich?}%
\label{sec:WissenschaftlichesSchreiben}

Wissenschaftliches Schreiben ist präzise, nüchtern, nachvollziehbar und überprüfbar.
Irreführende oder umständlich formulierte Aussagen sind genauso zu vermeiden wie Spannungsbögen oder Übertreibungen (z.B. \enquote{[...] ist ein herausragendes Ergebnis.}).
Wegen der oft komplexen Inhalte wird besonderer Wert auf eine einfache Beschreibung gelegt.

Gute wissenschaftliche Artikel sind sauber strukturiert und folgen dem \emph{Prinzip der Pyramide}:
Inhalte werden zunächst stark abstrahiert dargestellt und erst später im Detail erklärt, was das Verständnis des Lesers erhöhen soll.
Zudem weisen die meisten Papiere eine ähnliche Struktur auf: Zusammenfassung, Einleitung, Grundlagen, Hauptkapitel (z.B. der eigene Ansatz), Evaluation und Fazit.
Vor dem Fazit oder nach den Grundlagen wird Bezug auf verwandte Arbeiten genommen, da Forschung selten isoliert statt findet.
Jedes einzelne Kapitel wiederum sollte mit einem kurzen Überblick beginnen, gefolgt von detaillierteren Informationen.
Zu Zusammenfassung, Einleitung und Fazit gibt es darüber hinaus feste Vorgaben zu Inhalten und Reihenfolge.
Hierzu:

\smallskip
\begin{itemize}[label={\symbolInfo}]
    \item \href{https://sdqweb.ipd.kit.edu/wiki/Wissenschaftliches_Schreiben}{SDQ-Wiki --- Wissenschaftliches Schreiben }
    \item \href{https://sdqweb.ipd.kit.edu/wiki/Wissenschaftliches_Schreiben/Stil}{SDQ Wiki --- Wissenschaftliches Schreiben/Stil}
    \item \href{https://sdqweb.ipd.kit.edu/wiki/Wissenschaftliches_Schreiben/Aufbau}{SDQ Wiki --- Wissenschaftliches Schreiben/Aufbau}
\end{itemize}


\section{Recherchieren}%
\label{sec:Recherchieren}

Die Recherche, also das Suchen, Lesen und Zitieren anderer wissenschaftlicher Veröffentlichungen, gehört zu den Kernaufgaben wissenschaftlicher Arbeit.

\subsection{Richtig suchen}%
\label{sec:Recherchieren:Suchen}

Üblicherweise werden zu Beginn einer Seminar- oder Abschlussarbeit einige Quellen von den Betreuenden zur Verfügung gestellt.
Darüber hinaus sollen immer auch weitere Quellen und verwandte Arbeiten gesucht werden.
Zu den bekanntesten Werkzeugen hierfür zählen:

\smallskip
\begin{itemize}[label={\symbolTool}]
    \item \href{https://scholar.google.com/}{Google Scholar}, Suchmaschine für wissenschaftliche Veröffentlichungen aller Art
    \item \href{https://dblp.org/}{dblp}, eine umfassende Bibliographie der Informatik
    \item \href{https://citeseerx.ist.psu.edu/}{CiteSeer}, eine große und bekannte Zitationsdatenbank
\end{itemize}
\smallskip

\noindent
Mögliche Suchbegriffe sind Titel, Stichwörter, Autoren und Konferenzen (bzw. Kombinationen davon).
Diese können auch aus bereits vorhandener Literatur entnommen werden.
Eine weitere Möglichkeit ist das Folgen von Referenzen in vorliegender Literatur.
Suchmaschinen wie \toolLink{https://scholar.google.com/}{Google Scholar} ermöglichen zudem die Anzeige anderer Veröffentlichungen, die eine ausgewählte Quelle ebenfalls referenziert haben (\enquote{zitiert von: [...]}).
Viele Papiere sind frei verfügbar (\emph{Open Access}) oder lassen sich aus dem Uni-Netz bzw. über einen \toolLink{https://www.scc.kit.edu/dienste/vpn.php}{VPN} kostenlos abrufen.

Nicht alle Quellen haben die gleiche Wertigkeit. 
Präferiert sollte nach Journal- oder Konferenz-Artikeln (etwa von hochwertigen Publishern wie \emph{IEEE} oder \emph{ACM}) gesucht werden, da diese, wie auch Workshop-Papiere \emph{peer-reviewed} sind. 
Diese unabhängige Begutachtung durch andere Wissenschaftler gilt als wichtige Maßnahme der Qualitätssicherung. 
Zu Quellenarten ohne Peer-Review zählen technische Berichte, Dissertationen, Abschlussarbeiten oder Papiere in e-Print-Archiven wie \href{https://arXiv.org}{arXiv.org}.  
\href{https://www.wikipedia.org/}{Wikipedia} eignet sich nur für einen initialen Überblick und ist nicht zitierbar.

\subsection{Richtig lesen}%
\label{sec:Recherchieren:Lesen}

Papiere sollten stets mit einem Ziel oder definierten Fragen gelesen werden.
Um herauszufinden, ob ein gefundenes Papier relevant ist, reicht es nicht aus, nur den Titel zu lesen.
Zudem enthalten wissenschaftliche Veröffentlichungen eine Vielzahl von Details, die zur Beantwortung der eigenen Fragen nicht notwendig sind.
Aus diesen Gründen sollten Papiere \emph{zielgerichtet} gelesen werden: Zunächst Zusammenfassung, Gliederung, Einleitung und Zusammenfassung, und erst dann relevante Abschnitte.

Der Inhalt anderer Papiere sollte immer kritisch hinterfragt werden: Was wurde warum und wie gemacht, wie lassen sich Vorgehen, Ergebnisse und Evaluation bewerten?
Erkenntnisse und eigene Kommentare können mit farblichen Markierungen festgehalten werden, etwa für Kategorien wie: \emph{Relevant}, \emph{Unklar} oder \emph{Interessante Referenz}.

\subsection{Richtig zitieren}%
\label{sec:Recherchieren:Zitieren}

Alle Inhalte, die aus anderen Arbeiten stammen, müssen als Zitat gekennzeichnet werden.
In wissenschaftlichen Arbeiten werden Zitate üblicherweise an Ort und Stelle markiert und dann im Literaturverzeichnis am Ende des Dokuments aufgeführt.
Direkte Zitate sind wörtlich übernommen und werden in Anführungszeichen gefasst, indirekte Zitate werden z.B. am Ende der inhaltlich übernommenen Aussage markiert.
Das Erscheinungsbild der Markierungen ist meist vorgegeben, das \emph{IEEE} empfiehlt etwa Zahlen in eckigen Klammern \cite{IEEE2021}.
Hierzu:

\smallskip
\begin{itemize}[label={\symbolInfo}]
    \item \href{https://sdqweb.ipd.kit.edu/wiki/Zitieren}{SDQ-Wiki --- Zitieren}
\end{itemize}


\section{Schreiben}%
\label{sec:Schreiben}

Für die Erstellung wissenschaftlicher Dokumente aller Art hat sich das Textsatzsystem \TeX\ durchgesetzt.
Die Erweiterung \LaTeX\ ermöglicht es Autoren, sich auf den Inhalt ihrer Dokumente zu konzentrieren und weniger Fokus auf die Formatierung dieser legen zu müssen.

\subsection{Die Sprache \LaTeX}%
\label{sec:Schreiben:Sprache}

\LaTeX\ ist eine textbasierte Auszeichnungssprache (engl. \emph{Markup Language}) und folgt \emph{nicht} dem \emph{WYSIWYG}-Prinzip (\emph{\textbf{W}hat \textbf{Y}ou \textbf{S}ee \textbf{I}s \textbf{W}hat \textbf{Y}ou \textbf{G}et}) wie etwa Microsoft Word.
Dokumente werden mit Hilfe von Befehlen für Text-Formatierung, Überschriften, Tabellen, Listen, Zitaten, etc., erstellt und anschließend von der zugrundeliegenden \TeX-Engine übersetzt, üblicherweise in PDF-Dateien.
Hierzu:

\smallskip
\begin{itemize}[label={\symbolInfo}]
    \item \href{https://sdqweb.ipd.kit.edu/wiki/LaTeX}{SDQ-Wiki --- LaTeX}
    \item \href{https://learnxinyminutes.com/docs/de-de/latex-de/}{Lerne X in Y Minuten: LaTeX}
\end{itemize}

\subsection{Installation}%
\label{sec:Schreiben:Installation}

Um lokal \LaTeX-Dokumente erzeugen zu können, nutzt man eine Distribution wie \toolLink{https://www.tug.org/texlive/}{TeX Live}.
Diese bringt alle notwendigen Werkzeuge und eine Vielzahl an Paketen mit, die bei Bedarf eingebunden werden können, um die Basisfunktionalität z.B. um angepasste Tabellen, speziell formatierte Listen oder mathematische Symbole zu erweitern.

Zu den wichtigsten Dateitypen zählen: \fileType{tex} (\TeX-Dokumente), \fileType{bib} (Bibliographie), \fileType{cls} (Dokumentvorlagen) und \fileType{sty} (Stilvorlagen).
Beim Generieren von PDF-Dateien fallen zudem temporäre Dateien an; diese können meistens ignoriert werden.
Für einen besseren Überblick empfiehlt es sich, Inhalte in mehrere \fileType{tex}-Dateien aufzuteilen und diese mit Hilfe des \emph{Input}-Befehls einzubinden.

\subsection{Werkzeuge}%
\label{sec:Schreiben:Werkzeuge}

Um effizient \LaTeX-Dokumente schreiben zu können, wurden in den letzten Jahrzehnten eine Vielzahl von Editoren entwickelt. Viele davon sind, im Vergleich mit aktuellen IDEs, leider ziemlich in die Jahre gekommen. Daher empfehle ich:

\smallskip
\begin{itemize}[label={\symbolTool}]
    \item \href{https://code.visualstudio.com/}{Visual Studio Code}, eine Open-Source IDE, mit der \href{https://marketplace.visualstudio.com/items?itemName=James-Yu.latex-workshop}{LaTeX-Workshop}-Erweiterung für lokales Arbeiten
    \item \href{https://overleaf.com/}{Overleaf}, ein Online-Editor im Browser für kollaboratives Arbeiten (für KIT-Studierende ist die Premium-Version kostenlos, keine Installation notwendig)
\end{itemize}
\smallskip

\subsection{\bibtex\ vs. Bib\LaTeX}%
\label{sec:Schreiben:Bibtex}

\bibtex\ übernimmt die Verwaltung und Generierung des Literaturverzeichnisses, indem Referenzen im Text (z.B. mittels \emph{Cite}-Befehl) mit verlinkten \fileType{bib}-Dateien abgeglichen werden.
Bib\LaTeX\ erweitert diese Funktionalität unter anderem um Unicode-Support und verbesserte Anpassungsmöglichkeiten und bringt hierfür ein eigenes Backend namens \emph{biber} mit sich.
In Distributionen wie \toolLink{https://www.tug.org/texlive/}{TeX Live} sind beide enthalten.
Wenn möglich, sollte Bib\LaTeX\ auf Grund der moderneren Umgebung vorgezogen werden.

\subsection{Vorlagen}%
\label{sec:Schreiben:Vorlagen}

Eine große Stärke von \LaTeX\ ist die Trennung von Inhalt und Layout.
Während der Inhalt im Zentrum des wissenschaftlichen Schreibens steht, wird das Layout meist vorgegeben.
Dieses Dokument etwa nutzt die Vorlage von \emph{IEEE}-Papieren. Für Arbeiten am SDQ gibt es ebenfalls Vorlagen:

\smallskip
\begin{itemize}[label={\symbolInfo}]
    \item \href{https://sdqweb.ipd.kit.edu/wiki/Dokumentvorlagen}{SDQ-Wiki --- Dokumentvorlagen}
\end{itemize}

\subsection{Hilfreiche Pakete}%
\label{sec:Schreiben:Pakete}

Moderne \LaTeX-Distributionen bestehen aus mehreren Tausend Paketen, welche mittels \emph{usepackage}-Befehl eingebunden werden können.
Hilfreiche Dokumentationen zu diesen findet man auf \infoLink{https://ctan.org/}{CTAN}.
Um einen ersten Überblick zu bekommen, welche Pakete für eigene Dokumente hilfreich sein könnten, hier eine (unvollständige) Liste:

\smallskip
\begin{itemize}[label={\faCube}]
    \item \package{acronym} Verwaltet Abkürzungen und stellt sicher, dass diese (nur) einmal ausgeschrieben werden
    \item \package{amsmath} Fügt eine Vielzahl von Funktionen für mathematische Formeln hinzu
    \item \package{csquotes} Ermöglicht verschiedenste Zitationsstile, z.B. für die Unterstützung anderer Sprachen
    \item \package{babel} Fügt Sprachsupport für Schlüsselwörter in 250 Sprachen hinzu
    \item \package{booktabs} Hochwertige Tabellen, die den Anforderungen moderner Dokumente genügen
    \item \package{biblatex} Reimplementierung von \bibtex\ mit erweiterter Funktionalität, ist zusammen mit \emph{biber} zu nutzen
    \item \package{color} Ermöglicht erweiterte Einstellungen für Farben, z.B. für farbigen Text oder farbige Referenzen
    \item \package{enumitem} Erweitert die Konfigurierbarkeit von Listen, insbesondere die Kontrolle über Labels
    \item \package{graphicx} Fügt mehr Optionen für das  Einbinden von Grafiken hinzu
    \item \package{hyperref} Erleichtert (klickbare) Cross-Referenzen und Verlinkungen
    \item \package{listings} Automatisch formatierte Quelltext-Abschnitte, native Unterstützung von vielen Programmiersprachen
    \item \package{soul} Erweiterte Textformatierung inkl. Durchstreichen und farbiger Hervorhebung
    \item \package{stmaryrd} Beinhaltet viele (mathematische) Symbole und Klammern
    \item \package{todonotes} Ermöglicht die Hervorhebung von offenen Aufgaben und Problemen im Dokument selbst
\end{itemize}
\smallskip

\noindent
\emph{Hinweis}: Nur verwendete Pakete sollten importiert werden.
Zudem können manche Pakete auch bereits von der verwendeten Vorlage importiert und konfiguriert worden sein.

\subsection{Diagramme}%
\label{sec:Schreiben:Diagramme}

Abbildungen wie Diagramme bieten sich an, um komplexere Zusammenhänge wie Strukturen oder Abläufe zu visualisieren. 
Diagramme sollten dabei, wie auch Tabellen, nie losgelöst sein, sondern im Fließtext referenziert und erklärt werden. 
Zur Erstellung von Diagrammen existiert eine Vielzahl von Tools, darunter:

\smallskip
\begin{itemize}[label={\symbolTool}]
    \item \href{https://www.diagrams.net//}{diagrams.net}, auch bekannt als \emph{draw.io}, eine umfangreiche web-basierte Open-Source Software, kostenlos
    \item \href{https://www.lucidchart.com/}{Lucidchart}, eine proprietäre Plattform für Visualisierung aller Art (für Studierende kostenlos)
    \item \href{https://www.microsoft.com/de-de/microsoft-365/visio}{Microsoft Visio}, ein Teil von Microsoft 365 (für KIT-Studierende kostenlos)
\end{itemize}
\smallskip

\noindent
Alternativ können Diagramme auch direkt in \LaTeX\ mit Hilfe von \infoLink{https://sdqweb.ipd.kit.edu/wiki/TikZ}{TikZ} erstellt werden, was allerdings einen erheblichen Einarbeitungsaufwand mit sich bringt und für Anfänger gänzlich ungeeignet ist (eventuell gilt dasselbe auch für Fortgeschrittene...).

\section{Präsentieren}%
\label{sec:Praesentieren}

Nach der Fertigstellung einer wissenschaftlichen Ausarbeitung wird diese üblicherweise präsentiert -- das gilt sowohl für Seminararbeiten als auch für große Veröffentlichungen auf internationalen Konferenzen.

\subsection{Vortragen}%
\label{sec:Praesentieren:Vortragen}

Auf Vorträge treffen die gleichen Regeln und Tipps wie auch für Ausarbeitungen zu: Das \emph{Prinzip der Pyramide} beachten, an Vorgaben zu Vorlagen, Inhalt und Umfang halten, nüchtern und präzise präsentieren, Quellen sauber zitieren und kenntlich machen.
Darüber hinaus gelten auch die wohlbekannten Regeln eines guten Vortrags: Lesbare, nicht überladene Folien erstellen, Inhalte und insbesondere Grafiken erklären, frei und flüssig präsentieren, Backup-Folien für Rückfragen bereithalten. Hierzu:

\smallskip
\begin{itemize}[label={\symbolInfo}]
    \item \href{https://sdqweb.ipd.kit.edu/wiki/Vortragshinweise}{SDQ-Wiki --- Vortragshinweise}
\end{itemize}

\subsection{\LaTeX\ vs. Microsoft PowerPoint}%
\label{sec:Praesentieren:PowerPoint}

Studierenden ist es üblicherweise freigestellt, ob die Vortragsfolien mit \toolLink{https://ctan.org/pkg/beamer}{\LaTeX\ beamer} oder einem Präsentations-Tool wie PowerPoint erstellt werden.
Obwohl die Entscheidung für \LaTeX\ naheliegend scheint, hat dies auch Nachteile.
Die Erfahrung zeigt, dass mit \LaTeX\ erstellte Präsentationen von Studierenden nicht selten eine schlechtere Qualität u.a. bezüglich des Layouts aufweisen.
Im Gegensatz zu Dokumenten sind hier die technischen Hürden größer und gute Folien mit mehr Aufwand verbunden, weswegen die Entscheidung gut abgewägt werden sollte.


\section{Verwalten}%
\label{sec:Verwalten}

Wissenschaftliche Ausarbeitungen können sich über einen Zeitraum von wenigen Monaten (z.B. Seminararbeiten) bis hin zu vielen Jahren (z.B. Dissertationen) erstrecken.
Aus diesem Grund ist die saubere Verwaltung von Dokumenten und Literatur unumgänglich.

\subsection{Literaturverwaltung}%
\label{sec:Verwalten:Literatur}

Im Rahmen einer wissenschaftlichen Arbeit findet und ließt man eine Vielzahl anderer Veröffentlichungen.
Um hier den Überblick zu behalten, eignen sich Werkzeuge für die Literaturverwaltung wie das kostenlose \toolLink{https://www.zotero.org/}{Zotero}.
Zu den hilfreichen Funktionen solcher Werkzeuge zählen:

\smallskip
\begin{itemize}
    \item Verwalten und Organisieren von gefundener Literatur mittels Ordnern, Sammlungen, Tags und Notizen
    \item Automatisches Abrufen von Metadaten wie Titel, Autor oder Publisher aus Online-Datenbanken
    \item Einfaches Teilen der eigenen Bibliothek mit anderen Forschenden oder Studierenden
    \item Einfaches Hinzufügen neuer Literatur mittels Browser-Erweiterung
    \item Import und Export von \bibtex-Dateien
\end{itemize}

\subsection{Versionsverwaltung}%
\label{sec:Verwalten:Version}

Aktiv bearbeitete Dokumente wie Seminararbeiten und insbesondere Abschlussarbeiten sollten regelmäßig, im Idealfall täglich, gesichert werden.
Hierfür bietet sich die Versionsverwaltung mittels \emph{SVN} oder \emph{git} an.
Durch das regelmäßige Hochladen auf einen externen Server, wie z.B. \toolLink{https://git.scc.kit.edu/}{SCC GitLab} (kostenlos für KIT-Studierende), wird das Risiko des Datenverlusts minimiert.

Zudem ermöglicht \emph{git} das Laden früherer Versionen (sog. \emph{Commits}), was insbesondere bei technischen Problemen sehr hilfreich sein kann.
Da \LaTeX\- wie auch \bibtex\-Dateien textbasiert sind, lassen sich Änderungen einfach nachvollziehen und bei Bedarf rückgängig machen.
\emph{Hinweis}: Datenverlust ohne ausreichende Sicherungen ist explizit \emph{keine} Entschuldigung z.B. für Fristverlängerungen.
Kein Backup, kein Mitleid.
Erst recht als Informatiker.

\section{Weitere Tipps und Werkzeuge}%
\label{sec:Tipps}

Um dieses Dokument kurz zu halten, konnte eine Vielzahl weiterer nützlicher Tipps und Werkzeuge nicht mit aufgenommen werden. 
An dieser Stelle sei erneut auf das umfangreiche \infoLink{https://sdqweb.ipd.kit.edu/wiki/Wissenschaftliches_Schreiben}{SDQ-Wiki} verwiesen. 
Ebenfalls hilfreich:

\smallskip
\begin{itemize}[label={\faLightbulbO}]
    \item Es gibt viele kostenlose aber mächtige Übersetzer-Tools wie \toolLink{https://www.deepl.com/translator}{DeepL} und \toolLink{https://www.linguee.de/}{Linguee}
    \item Microsoft Word kann zur Sprach- und Grammatik-Prüfung eingesetzt werden. Mit \LaTeX\ generierte PDF-Dateien lassen sich hierzu direkt in Word öffnen
    \item Bei Änderungen in Referenzen oder der Bibliographie müssen zunächst \fileType{tex}-Dateien übersetzt, dann \bibtex\ bzw. \emph{biber} ausgeführt und dann nochmals übersetzt werden. Diesen Vorgang automatisiert \toolLink{https://ctan.org/pkg/latexmk}{latexmk}
    \item Die allermeisten Fragen zu \LaTeX\ wurden schon mal gestellt und z.B. auf dem \infoLink{https://tex.stackexchange.com/}{\TeX\ StackExchange} beantwortet
\end{itemize}
